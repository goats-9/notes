\documentclass[journal,12pt,twocolumn]{IEEEtran}
\usepackage{setspace}
\usepackage{gensymb}
\usepackage{xcolor}
\usepackage{caption}
\singlespacing
\usepackage{siunitx}
\usepackage[cmex10]{amsmath}
\usepackage{mathtools}
\usepackage{hyperref}
\usepackage{amsthm}
\usepackage{mathrsfs}
\usepackage{txfonts}
\usepackage{stfloats}
\usepackage{cite}
\usepackage{cases}
\usepackage{subfig}
\usepackage{longtable}
\usepackage{multirow}
\usepackage{enumitem}
\usepackage{mathtools}
\usepackage{listings}
\usepackage{tikz}
\usetikzlibrary{shapes,arrows,positioning}
\usepackage{circuitikz}
\usepackage{tabularx}
\let\vec\mathbf
\DeclareMathOperator*{\Res}{Res}
\renewcommand\thesection{\arabic{section}}
\renewcommand\thesubsection{\thesection.\arabic{subsection}}
\renewcommand\thesubsubsection{\thesubsection.\arabic{subsubsection}}

\renewcommand\thesectiondis{\arabic{section}}
\renewcommand\thesubsectiondis{\thesectiondis.\arabic{subsection}}
\renewcommand\thesubsubsectiondis{\thesubsectiondis.\arabic{subsubsection}}
\hyphenation{op-tical net-works semi-conduc-tor}

\lstset{
language=sql,
frame=single, 
breaklines=true,
columns=fullflexible
}
\begin{document}
\theoremstyle{definition}
\newtheorem{theorem}{Theorem}[section]
\newtheorem{problem}{Problem}
\newtheorem{proposition}{Proposition}[section]
\newtheorem{lemma}{Lemma}[section]
\newtheorem{corollary}[theorem]{Corollary}
\newtheorem{example}{Example}[section]
\newtheorem{definition}{Definition}[section]
\newcommand{\BEQA}{\begin{eqnarray}}
\newcommand{\EEQA}{\end{eqnarray}}
\newcommand{\define}{\stackrel{\triangle}{=}}
\newcommand{\myvec}[1]{\ensuremath{\begin{pmatrix}#1\end{pmatrix}}}
\newcommand{\mydet}[1]{\ensuremath{\begin{vmatrix}#1\end{vmatrix}}}
\bibliographystyle{IEEEtran}
\providecommand{\nCr}[2]{\,^{#1}C_{#2}} % nCr
\providecommand{\nPr}[2]{\,^{#1}P_{#2}} % nPr
\providecommand{\mbf}{\mathbf}
\providecommand{\pr}[1]{\ensuremath{\Pr\left(#1\right)}}
\providecommand{\qfunc}[1]{\ensuremath{Q\left(#1\right)}}
\providecommand{\sbrak}[1]{\ensuremath{{}\left[#1\right]}}
\providecommand{\lsbrak}[1]{\ensuremath{{}\left[#1\right.}}
\providecommand{\rsbrak}[1]{\ensuremath{{}\left.#1\right]}}
\providecommand{\brak}[1]{\ensuremath{\left(#1\right)}}
\providecommand{\lbrak}[1]{\ensuremath{\left(#1\right.}}
\providecommand{\rbrak}[1]{\ensuremath{\left.#1\right)}}
\providecommand{\cbrak}[1]{\ensuremath{\left\{#1\right\}}}
\providecommand{\lcbrak}[1]{\ensuremath{\left\{#1\right.}}
\providecommand{\rcbrak}[1]{\ensuremath{\left.#1\right\}}}
\theoremstyle{remark}
\newtheorem{rem}{Remark}
\newcommand{\sgn}{\mathop{\mathrm{sgn}}}
\newcommand{\rect}{\mathop{\mathrm{rect}}}
\newcommand{\sinc}{\mathop{\mathrm{sinc}}}
\providecommand{\abs}[1]{\left\vert#1\right\vert}
\providecommand{\res}[1]{\Res\displaylimits_{#1}} 
\providecommand{\norm}[1]{\lVert#1\rVert}
\providecommand{\mtx}[1]{\mathbf{#1}}
\providecommand{\mean}[1]{E\left[ #1 \right]}
\providecommand{\fourier}{\overset{\mathcal{F}}{ \rightleftharpoons}}
\providecommand{\ztrans}{\overset{\mathcal{Z}}{ \rightleftharpoons}}
\providecommand{\system}[1]{\overset{\mathcal{#1}}{ \longleftrightarrow}}
\newcommand{\solution}{\noindent \textbf{Solution: }}
\providecommand{\dec}[2]{\ensuremath{\overset{#1}{\underset{#2}{\gtrless}}}}
\numberwithin{equation}{section}
\makeatletter
\@addtoreset{figure}{problem}
\makeatother
\let\StandardTheFigure\thefigure
\renewcommand{\thefigure}{\theproblem}
\def\putbox#1#2#3{\makebox[0in][l]{\makebox[#1][l]{}\raisebox{\baselineskip}[0in][0in]{\raisebox{#2}[0in][0in]{#3}}}}
     \def\rightbox#1{\makebox[0in][r]{#1}}
     \def\centbox#1{\makebox[0in]{#1}}
     \def\topbox#1{\raisebox{-\baselineskip}[0in][0in]{#1}}
     \def\midbox#1{\raisebox{-0.5\baselineskip}[0in][0in]{#1}}
\vspace{3cm}
\title{Chapter 2: Operating System Structures}
\maketitle
\tableofcontents
\renewcommand{\thefigure}{\theenumi}
\renewcommand{\thetable}{\theenumi}
\bigskip

\section{Operating System Services}
\begin{enumerate}
    \item \textbf{User Interface:}
    \begin{enumerate}
        \item Most OSes have a \textbf{user interface} (UI), most commonly a
        \textbf{graphical user interface} (GUI).
        \item In GUIs, the interface is a window system with a pointing device to 
        click and a keyboard to enter twxt.
        \item Mobile phones and tables have a \textbf{touchscreen interface},
        enabling selection and typing with fingers.
        \item A \textbf{command line interface} (CLI) users text commands only.
        \item Some systems may provide more than one of these variations.
    \end{enumerate}
    \item \textbf{Program execution:} The system must be able to load and run 
    a program. The program must terminate, either normally or abnormally, 
    raising an error.
    \item \textbf{I/O Operations:} A running program may require I/O. For 
    efficiency and protection, users cannot directly control I/O devices. OS
    provides a means to do I/O.
    \item \textbf{File system manipulation:}
    \begin{enumerate}
        \item Programs need to be able to create/delete files, read/write files
        and search by name.
        \item OSes include permissions on management and ownership of files.
        \item Many OSes provide a variety of file systems offering specific
        features and performance characteristics.
    \end{enumerate}
    \item \textbf{Communications:}
    \begin{enumerate}
        \item Processes may need to exchange information with each other. These 
        processes may be executing on the same computer or networked systems.
        \item \textbf{Shared memory} involves processes reading and writing to
        a shared section of memory.
        \item \textbf{Message parsing} involves packets of information in 
        predefined formats that are moved by processes.
    \end{enumerate}
    \item \textbf{Error detection:}
    \begin{enumerate}
        \item OS must be able to detect and correct errors. 
        \item These may occur in CPU and memory hardware (memory error or power 
        failure), I/O devices (connection error, parity error) or in the user 
        program (arithmetic overflow, illegal instruction).
        \item The OS must take appropriate action to ensure correct and 
        consistent computing. It may have to halt the system, or could return an 
        error code.
    \end{enumerate}
    \item \textbf{Resource allocation:}
    \begin{enumerate}
        \item Resources must be allocated to multiple processes running at the 
        same time. OS may manage different types of resources.
        \item Some resources may have specific allocation code, while others may
        have a more general request code. There may be other routines for 
        peripheral devices.
        \item CPU-scheduling routines take into account CPU speed, number of 
        queued processes and so on while allocating resources. 
    \end{enumerate}
    \item \textbf{Logging:}
    \begin{enumerate}
        \item Used to keep track of which processes use what resources.
        \item May be used for billing, accumulating statistics and configuring 
        the system to enhance performance.
    \end{enumerate}
    \item \textbf{Protection and security:}
    \begin{enumerate}
        \item Owners of information may want to control its access in multiuser or 
        networked system.
        \item Concurrently executing processes should not be able to interfere with 
        each other.
        \item Protection involves ensuring that access to system resources is 
        controlled.
        \item Security involves preventing outside users access to the system. This
        involves password authentication and recording I/O devices for instances
        of invalid access or break-ins.
    \end{enumerate}
\end{enumerate}

\section{User and Operating System Interface}
\subsection{Command Interpreters}
\begin{enumerate}
    \item A \textbf{command interpreter} allows users to directly enter commands 
    to be performed by the OS.
    \item Most OSes treat the command interpreter as a special program when a user
    logs on or a process starts.
    \item OSes that provide mulitple command interpretes call them 
    \textbf{shells}. For example, \textit{Bourne Again Shell} (\texttt{bash}) 
    or \textit{Korn Shell}.
    \item Main function of command interpreter is to get and execute the next
    user command. A common approach is to load a program and execute it with
    arguments.
\end{enumerate}

\subsection{Graphical User Interface}
\begin{enumerate}
    \item \textbf{Icons} represent programs, files, functions and directories.
    \item Clicking with the mouse can open menus or directories, called 
    \textbf{folders}.
    \item Examples of GUIs for UNIX systems include KDE, GNOME, etc.
\end{enumerate}

\subsection{Touch Screen Interface}
\begin{enumerate}
    \item Command-line interface or mouse and keyboard system is impractical 
    for mobile phones. Touchscreen interface used instead.
    \item Interaction is made by using \textbf{gestures} on the touch screen.
    \item Keyboards are simulated on the touch screen.
\end{enumerate}

\subsection{Choice of Interface}
\begin{enumerate}
    \item \textbf{System administrators} and \textbf{power users} who know systems 
    use the CLI, as it is more efficient and faster access, and makes repetitive
    tasks easier, partly due to their own programmability (\textbf{shell scripts}).
    \item Casual users use GUIs for tasks such as browsing the web or watching
    videos.
    \item User interfaces may vary between systems and also between users within a 
    system.
\end{enumerate}

\section{System Calls}
\begin{enumerate}
    \item \textbf{System calls} provide an interface to the services made 
    available by an OS.
    \item Generally low level functions implemented in C/C++/Assembly.
\end{enumerate}

\subsection{Application Programming Interface}
\begin{enumerate}
    \item Application developers design programs according to an
    \textbf{application programming interface} (API). The API specifies a set
    of functions that are available to an application programmer, along with 
    parameters to pass and return values to expect.
    \item The \textbf{libc} library written in C for UNIX OSes is an API to
    access OS code. These API functions typically implement actual system 
    calls.
    \item Benefits of transferring control to an API include portability
    and ease of use.
    \item The \textbf{run time environment} (RTE) is a software suite needed to
    execute applications in a programming language. RTEs provide a 
    \textbf{system call interface} that links system calls to the API functions.
    \item Programmers must only obey the API, while most of the OS interface
    is hidden from them.
    \item Three methods used to pash parameters to the OS:
    \begin{enumerate}
        \item Using registers (at most five parameters).
        \item Storing parameters in a block or a table and storing the memory 
        address of the block in the register.
        \item Using \textbf{stacks} to \textbf{push} or \textbf{pop} parameters.
    \end{enumerate}
\end{enumerate}

\subsection{Types of System Calls}
\subsubsection{Process Control}
\begin{enumerate}
    \item If a system call terminates a program abnormally, the memory of the 
    system at that point is dumped and an error message shown.
    \item Memory dumps may be examined using a \textbf{debugger} to find errors,
    or \textbf{bugs} in a program.
    \item Users interact with the command interpreter to issue appropriate 
    actions in case of error.
    \item Some systems may allow special recovery actions in case of error.
    \item Error levels may be defined to distinguish severity of errors.
    \item Processes should be able to create new child processes and access 
    their attributes.
    \item OS must manage concurrent access of a shared resource, usually via
    \textbf{locks}.
\end{enumerate}

\subsubsection{File Management}
\begin{enumerate}
    \item System calls are required to create/delete files.
    \item Files should be opened, read/written/stream repositioned and closed 
    via system calls.
    \item System calls should also provide for accessing attributes of the 
    file.
    \item System calls can be provided for copying and moving files.
\end{enumerate}

\subsubsection{Device Management}
\begin{enumerate}
    \item Various resources controlled by OS can be thought of as devices. They 
    can be either physical or virtual.
    \item A system should request for exclusive access and be able to release a
    device.
    \item When the device is allocated, system should be able to read, write and
    reposition the device, just like files.
\end{enumerate}

\subsubsection{Information Maintenance}
\begin{enumerate}
    \item Includes system calls for current timestamp and system information.
    \item System calls to dump memory and print stacktrace.
    \item Microprocessors provide \textbf{single step} mode, where the CPU 
    executes a trap caught by a debugger.
    \item System calls to provide time profiles of a program; these indicate the 
    amount of time a program executes at a set of locations.
    \item System calls to get and set process attributes.
\end{enumerate}

\subsubsection{Communication}
\begin{enumerate}
    \item \textbf{Message-passing model:}
    \begin{enumerate}
        \item Communicating processes exchange messages with one another to
        transfer information.
        \item Communication may be direct or indirect through a mailbox.
        \item A connection must be opened to communicate. 
        \item \textbf{Host names} or IP addresses of computers in a network or 
        \textbf{process names} of various processes within a computer must be 
        translated into known form for the OS to refer.
        \item Processes that receive connections are called \textbf{daemons}, 
        which are system programs for this purpose.
        \item The source of communication is known as the \textbf{client} and 
        the receiveing daemon is known as the \textbf{server}.
        \item System calls are provided for translation of names, managing 
        connections and for sending messages.
    \end{enumerate}
    \item \textbf{Shared-memory model:}
    \begin{enumerate}
        \item Information exchanged by reading and writing in shared memory 
        areas between processes.
        \item Processes must ensure that they are not writing simultaneously 
        to the shared location.
        \item System calls are used to open and close shared memory regions.
    \end{enumerate}
    \item Message passing is useful for smaller amounts of data, and easier to
    implement than shared memory.
    \item Shared memory allows for maximum speed and convenience of communication.
    However, problems may exist in protection and synchronization of processes.
\end{enumerate}

\subsubsection{Protection}
\begin{enumerate}
    \item Provides a mechanism to control access to resources.
    \item System calls can set permissions of files and disks, or
    prevent certain users from accessing resources.
\end{enumerate}

\section{System Services}
\begin{enumerate}
    \item \textbf{System services} or \textbf{System utilities} provide a 
    convenient environment for program development and execution.
    \item The various types of system services are:
    \begin{enumerate}
        \item \textbf{File management:} Perform operations such as create, 
        delete, list, access, rename, etc. on files and directories.
        \item \textbf{Status information:} Provide system time and resources;
        can be more complex, providing detailed performance, logging and 
        debugging information. Some provide a \textbf{registry} to store and 
        retrieve configuration information.
        \item \textbf{File modification:} Text editors may create and modify 
        content of files, or search and perform transformations on the text.
        \item \textbf{Programming language support:} Compilers, interpreters,
        assemblers and debuggers provided to create programs in high-level
        programming languages.
        \item \textbf{Program loading and execution:} Loaders, linkers, and 
        to execute and debug a program.
        \item \textbf{Communications:} These programs create connections among processes, 
        users and systems. Allow for sending messages, file transfer or logging 
        in remotely.
        \item \textbf{Background services:} System-program processes called 
        daemons, \textbf{services} or \textbf{subsystems}. For example, daemons 
        to manage remote connections, process schedulers, system monitoring error 
        services, etc.
    \end{enumerate}
    \item \textbf{Application programs} such as web browsers, e-mail clients, 
    video/image viewers, word processors, etc. useful for users to solve 
    everyday problems. 
\end{enumerate}

\section{Linkers and Loaders}
\begin{enumerate}
    \item Source files are compiled into \textbf{relocatable object files}, that 
    can be loaded into any physical location in memory.
    \item A \textbf{linker} combines the relocatable object files into a single
    binary \textbf{executable} file. Other library object files may be included
    in this phase as well.
    \item A \textbf{loader} loads binary executable files into memory 
    locations where eligible to run on a CPU core.
    \item Associated with linking and loading is \textbf{relocation}, which 
    refers to assigning the final addresses of program parts and adjusting code
    and data to match those addresses so that library functions and variables 
    can be accessed during execution.
    \item When an executable is to be run, the shell forks a process for it, 
    then invokes the loader to load the specified program into memory using the 
    address space of the process.
    \item Systems allow a program to dynamically link libraries when being 
    loaded. These are known in Windows as dynamically linked libraries 
    (\textbf{DLLs}).
    \item Advantages of dynamic linking are that unnecessary libraries are not 
    linked and loaded, and multiple processes can save memory use by sharing DLLs.
    \item In UNIX systems, object files and executables are stored in
    \textbf{Executable and Linkable Format} (ELF). There are seaprate ELF formats 
    for executable and relocatable files.
    \item An ELF executable contains the program's entry point, which contains 
    the first instruction that the program runs.
    \item Windows uses the \textbf{Portable Executable} (PE) format and macOS 
    uses the \textbf{Mach-O} format.   
\end{enumerate}

\section{Why Applications are Operating System Specific}
\begin{enumerate}
    \item An application can be made cross-platform using these approaches:
    \begin{enumerate}
        \item The application is written in an interpreted language such as 
        Python or Ruby, which has an interpreter for multiple OSes. 
        Performance is limited compared to the native OS, and the interpreter 
        provides a subset of the features of the OS, limiting feature sets of 
        the applications. 
        \item The application can be written in an language that requires a
        virtual machine for execution, for example Java. It has an RTE, 
        including a loader, byte-code verifier, etc. ported to various OSes 
        to load and run the application in the virtual machine. Same 
        disadvantages as for interpreters.
        \item Application is written in an OS specific API, hence it must be 
        separately ported to other OSes entirely. This is cumbersome and must 
        be done with each update. 
    \end{enumerate}
    \item APIs are not provided across platforms, making it difficult to run 
    applications on other OSes.
    \item Each OS has a binary format for its applications. This includes 
    header layouts, instructions and variables, and their locations in the file.
    \item CPUs have varying instruction sets. Applications with the appropriate
    instruction sets will run properly.
    \item System calls used by applications in one OS may not be present in 
    another OS.
    \item UNIX systems have adopted ELF formats to help this, but ELF is not 
    tied to any CPU architecture.
    \item \textbf{Application Binary Interface} (ABI) defines how different 
    components of binary code can interface for a given OS and CPU architecture.
    \item ABIs specify low-level details such as address width, methods of passing
    arguments to functions, size of data types, etc.
    \item An ABI is specified for a given architecture and OS, meaning it does
    little for cross-platform compatability.
\end{enumerate}

\section{Operating System Design and Inplementation}
\subsection{Design Goals}
\begin{enumerate}
    \item These can be grouped as \textbf{user goals} and \textbf{system goals}.
    \item Users want ease of use, ease of learning, reliability, speed and 
    safety.
    \item Developers want flexible and easy implementation which is error free 
    and reliable.
    \item There is no unique solution for defining the goals and they vary with 
    the OS. This is a creative task.
    \item The field of \textbf{software engineering} develops general design 
    principles for creating an OS.
\end{enumerate}

\subsection{Mechanisms and Policies}
\begin{enumerate}
    \item \textbf{Mechanisms} determine how to do something.
    \item \textbf{Policies} determine what will be done.
    \item 
\end{enumerate}
\end{document}