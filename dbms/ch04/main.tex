\documentclass[journal,12pt,twocolumn]{IEEEtran}
\usepackage{setspace}
\usepackage{gensymb}
\usepackage{xcolor}
\usepackage{caption}
\singlespacing
\usepackage{siunitx}
\usepackage[cmex10]{amsmath}
\usepackage{mathtools}
\usepackage{hyperref}
\usepackage{amsthm}
\usepackage{mathrsfs}
\usepackage{txfonts}
\usepackage{stfloats}
\usepackage{cite}
\usepackage{cases}
\usepackage{subfig}
\usepackage{longtable}
\usepackage{multirow}
\usepackage{enumitem}
\usepackage{mathtools}
\usepackage{listings}
\usepackage{tikz}
\usetikzlibrary{shapes,arrows,positioning}
\usepackage{circuitikz}
\usepackage{tabularx}
\let\vec\mathbf
\DeclareMathOperator*{\Res}{Res}
\renewcommand\thesection{\arabic{section}}
\renewcommand\thesubsection{\thesection.\arabic{subsection}}
\renewcommand\thesubsubsection{\thesubsection.\arabic{subsubsection}}

\renewcommand\thesectiondis{\arabic{section}}
\renewcommand\thesubsectiondis{\thesectiondis.\arabic{subsection}}
\renewcommand\thesubsubsectiondis{\thesubsectiondis.\arabic{subsubsection}}
\hyphenation{op-tical net-works semi-conduc-tor}

\lstset{
language=sql,
frame=single, 
breaklines=true,
columns=fullflexible
}
\begin{document}
\theoremstyle{definition}
\newtheorem{theorem}{Theorem}[section]
\newtheorem{problem}{Problem}
\newtheorem{proposition}{Proposition}[section]
\newtheorem{lemma}{Lemma}[section]
\newtheorem{corollary}[theorem]{Corollary}
\newtheorem{example}{Example}[section]
\newtheorem{definition}{Definition}[section]
\newcommand{\BEQA}{\begin{eqnarray}}
\newcommand{\EEQA}{\end{eqnarray}}
\newcommand{\define}{\stackrel{\triangle}{=}}
\newcommand{\myvec}[1]{\ensuremath{\begin{pmatrix}#1\end{pmatrix}}}
\newcommand{\mydet}[1]{\ensuremath{\begin{vmatrix}#1\end{vmatrix}}}
\bibliographystyle{IEEEtran}
\providecommand{\nCr}[2]{\,^{#1}C_{#2}} % nCr
\providecommand{\nPr}[2]{\,^{#1}P_{#2}} % nPr
\providecommand{\mbf}{\mathbf}
\providecommand{\pr}[1]{\ensuremath{\Pr\left(#1\right)}}
\providecommand{\qfunc}[1]{\ensuremath{Q\left(#1\right)}}
\providecommand{\sbrak}[1]{\ensuremath{{}\left[#1\right]}}
\providecommand{\lsbrak}[1]{\ensuremath{{}\left[#1\right.}}
\providecommand{\rsbrak}[1]{\ensuremath{{}\left.#1\right]}}
\providecommand{\brak}[1]{\ensuremath{\left(#1\right)}}
\providecommand{\lbrak}[1]{\ensuremath{\left(#1\right.}}
\providecommand{\rbrak}[1]{\ensuremath{\left.#1\right)}}
\providecommand{\cbrak}[1]{\ensuremath{\left\{#1\right\}}}
\providecommand{\lcbrak}[1]{\ensuremath{\left\{#1\right.}}
\providecommand{\rcbrak}[1]{\ensuremath{\left.#1\right\}}}
\theoremstyle{remark}
\newtheorem{rem}{Remark}
\newcommand{\sgn}{\mathop{\mathrm{sgn}}}
\newcommand{\rect}{\mathop{\mathrm{rect}}}
\newcommand{\sinc}{\mathop{\mathrm{sinc}}}
\providecommand{\abs}[1]{\left\vert#1\right\vert}
\providecommand{\res}[1]{\Res\displaylimits_{#1}} 
\providecommand{\norm}[1]{\lVert#1\rVert}
\providecommand{\mtx}[1]{\mathbf{#1}}
\providecommand{\mean}[1]{E\left[ #1 \right]}
\providecommand{\fourier}{\overset{\mathcal{F}}{ \rightleftharpoons}}
\providecommand{\ztrans}{\overset{\mathcal{Z}}{ \rightleftharpoons}}
\providecommand{\system}[1]{\overset{\mathcal{#1}}{ \longleftrightarrow}}
\newcommand{\solution}{\noindent \textbf{Solution: }}
\providecommand{\dec}[2]{\ensuremath{\overset{#1}{\underset{#2}{\gtrless}}}}
\numberwithin{equation}{section}
\makeatletter
\@addtoreset{figure}{problem}
\makeatother
\let\StandardTheFigure\thefigure
\renewcommand{\thefigure}{\theproblem}
\def\putbox#1#2#3{\makebox[0in][l]{\makebox[#1][l]{}\raisebox{\baselineskip}[0in][0in]{\raisebox{#2}[0in][0in]{#3}}}}
     \def\rightbox#1{\makebox[0in][r]{#1}}
     \def\centbox#1{\makebox[0in]{#1}}
     \def\topbox#1{\raisebox{-\baselineskip}[0in][0in]{#1}}
     \def\midbox#1{\raisebox{-0.5\baselineskip}[0in][0in]{#1}}
\vspace{3cm}
\title{Chpater 4: Intermediate SQL}
\maketitle
\tableofcontents
\renewcommand{\thefigure}{\theenumi}
\renewcommand{\thetable}{\theenumi}
\bigskip

\section{Join Expressions}

\textbf{Joins} are used to combine information from two or more relations. They 
can be of various types.

\subsection{The Natural Join}
\begin{enumerate}
    \item The \textbf{natural join} operation operates on two relations and 
    produces a subset of the Cartesian product of those relations which contains 
    tuples whose values of attributes that appear in the schema of both 
    relationships is the \textit{same}.
    \item SQL syntax:
    \begin{lstlisting}
    table1 natural join table2;
    \end{lstlisting}
    \item Note that while joining more than two relations, the order of joining 
    matters. SQL provides a \textbf{join ... using} construct to specify which 
    columns should be updated. Syntax:
    \begin{lstlisting}
    table1 join table2 using (attr_1, attr_2, ..., attr_n);
    \end{lstlisting}
\end{enumerate}

\subsection{Join Conditions}
\begin{enumerate}
    \item The \textbf{join ... using} clause requires values to match on 
    specified attributes.
    \item To join relations based on general conditions, the 
    \textbf{join ... on} clause is used. Syntax:
    \begin{lstlisting}
    table1 join table2 on pred;
    \end{lstlisting}
    \item An equivalent expression if the \textbf{on} clause is not used is to 
    move the predicate to the \textbf{where} clause.
    \item However, the reasons for using an \textbf{on} condition are:
    \begin{enumerate}
        \item It behaves differently as compared to \textbf{where} clauses on 
        outer joins.
        \item The join condition is specified separately. This makes the query 
        readable by humans.
    \end{enumerate}
\end{enumerate}

\subsection{Outer Joins}
\begin{enumerate}
    \item Works similar to other joins, but preserves lost tuples by creating 
    tuples in the result containing null values.
    \item They are of three forms:
    \begin{enumerate}
        \item \textbf{left outer join:} Preserves tuples only in the relation 
        named before the operation.
        \item \textbf{right outer join:} Preserves tuples only in the relation 
        named after the operation.
        \item \textbf{full outer join:} Preserves tuples in both relations.
    \end{enumerate}
    \item Syntax similar to the inner join syntax.
    \item To compute outer join:
    \begin{enumerate}
        \item Compute inner join.
        \item For every tuple $t$ in the relation (preceding/following) the 
        operation that does not match any tuple in the other relation, add a 
        tuple $r$ consisting of values of $t$ with the remaining values padded 
        by \textbf{null}.
        \item A full outer join is the union of left and right outer join.
        \item \textbf{on} clause can be used instead of \textbf{natural} as 
        before.
        \item \textbf{on} and \textbf{where} behave differently for outer 
        joins. The \textbf{on} clause is part of the outer join specification,
        while the \textbf{where} clause is applied after performing the join.
    \end{enumerate}
\end{enumerate}

\subsection{Join Types and Conditions}

Normal joins are called \textbf{inner joins} in SQL. An inner join is the 
default behaviour when \textbf{join} is specified. Similarly, 
\textbf{natural join} is equivalent to \textbf{natural inner join}.

\section{Views}
\begin{enumerate}
    \item Not always desirable for all users to see the entire set of relations 
    in a database.
    \item May require a personalized collection of relations to match a certain 
    user's intention of what the database looks like.
    \item A \textbf{view} is a \textit{virtual relation} defined by a query and 
    the relation contains the result of the query. It is not stored but computed 
    whenever it is used from the existing relations.
\end{enumerate}

\subsection{View Definition}
\begin{enumerate}
    \item Use the \textbf{create view} command. Syntax:
    \begin{lstlisting}
    create view tableName(viewSchema) as viewName;
    \end{lstlisting}
    \item Difference from \textbf{with} clause is that views remain available 
    until explicity dropped, while the relations defined in the \textbf{with} 
    clause are local to that query only.
\end{enumerate}

\subsection{Using Views in SQL Queries}

Can be used wherever an actual relation appears in a query statement. One can 
define views using other existing views also.

\subsection{Materialized Views}
\begin{enumerate}
    \item Views that are stored but updated when the underlying relations are 
    updated are called \textbf{materialized views}.
    \item The process of keeping a materialized view up to date is called 
    \textbf{materialized view maintenance}. It can be immediate, lazy, or 
    periodic.
    \item Materialized views may be favourable if the view is frequently used,
    since it leads to faster response times. However, there is a tradeoff in 
    storage and added overhead for updates.
\end{enumerate}
  
\subsection{Update of a View}
\begin{enumerate}
    \item Difficulties of updating a view:
    \begin{enumerate}
        \item Modifications to the view must be translated to modifications of 
        the database.
        \item Other attributes not included the view must be filled with nulls.
    \end{enumerate}
    \item An SQL view is said to be \textbf{updatable} if:
    \begin{enumerate}
        \item The \textbf{from} clause has only one relation.
        \item The \textbf{select} clause contiains only the attribute names of 
        the relation, and not any aggregated, expressions, or \textbf{distinct}
        specification.
        \item Any attribute \textit{not} listed in the \textbf{select} clause 
        should be set to \textbf{null}.
        \item The query defining the view does \textit{not} have a 
        \textbf{group by} or \textbf{having} clause.
    \end{enumerate}
    \item Views can be defined with the \textbf{with check option} clause, 
    meaning any insertion or update not satisfying the \textbf{where} clause 
    condition of the view is rejected by the database.
    \item A preferred approach is to use trigger mechanism. The \textbf{instead 
    of} feature replaces default update, insert and delete mechanisms with 
    actions designed specifically for each case.
\end{enumerate}

\section{Transactions}
\begin{enumerate}
    \item A transaction consists of a sequence of query and/or update statements.
    \item A transaction begins implicitly when an SQL statement is executed.
    \item Must be ended by one of:
    \begin{enumerate}
        \item 
        \item \textbf{commit (work):} Commits the current transaction; makes 
        the updates permanent in the database. A new transaction is started 
        after commit.
        \item \textbf{rollback (work):} Rolls back the current transaction; 
        undoes all updates performed by the transaction. Database restored to 
        the state it was in prior to the execution of the first statement of 
        the transaction. Cannot undo effects of already committed transactions.
    \end{enumerate}
    \item Database provides abstraction of transaction as being \textbf{atomic}.
    \item Many SQL implementations consider each update statement as a single 
    transaction and commit after each action. This \textit{auto-commit} feature 
    can be disabled by typing \textbf{set autocommit off}.
    \item A better alternative to this is to enclose statements that make up a 
    transaction in \textbf{begin atomic ... end} as in SQL:1999. Some systems, 
    however, use \textbf{commit/rollback (work)} instead of \textbf{end} as 
    above.
\end{enumerate}

\section{Integrity Constraints}
\begin{enumerate}
    \item Ensure that changes made to the database by authorized users do not 
    result in a loss of data consistency. Note that constraints on authorization 
    are called \textit{security constraints}.
    \item They are specified in the DDL, either when creating a table, or when 
    altering a table.
\end{enumerate}

\subsection{Not Null Constraint}
\begin{enumerate}
    \item Prohibits insertion of null value for the attribute. Example of 
    \textbf{domain constraint}.
    \item SQL prohibits insertion of null values in primary keys of a relation.
\end{enumerate}
  
\subsection{Unique Constraint}
\begin{enumerate}
    \item Syntax:
    \begin{lstlisting}
    unique (attr_1, attr_2, ..., attr_n);
    \end{lstlisting}
    \item Specifies that the attributes in the brackets form a \textit{superkey}.
    \item Note that attributes declared to be unique \textit{can} be \textbf{null}.
\end{enumerate}
  
\subsection{The Check Clause}
\begin{enumerate}
    \item Syntax: \textbf{check}($P$), where $P$ is a predicate to be satisfied
    by every tuple in the relation.
    \item A check is \textit{satisfied} if it is NOT \textbf{false}, hence 
    tuples returning \textbf{unknown} are not violations. A \textbf{not null} 
    constraint can eliminate such nulls.
    \item A check can appear in the schema as part of the declaration of an 
    attribute, or at the end of the declaration of the schema.
    \item According to the SQL standard, $P$ can contain a predicate, though it 
    is not widely implemented in popular database systems.
\end{enumerate}
  
\subsection{Referential Integrity}
\begin{enumerate}
    \item Foreign keys can be specified using the \textbf{foreign key} clause.
    \item Syntax:
    \begin{lstlisting}
    foreign key forKey references tableName(schema);    
    \end{lstlisting}
    Here, \texttt{forKey} is compatible with the primary key of the referenced 
    relation.
    \item Can specify what happens if the referenced relation is changed, by
    specifying \textbf{on (delete | update) (cascade | set null | set $v$)} in 
    the declaration of the foreign key constraint. This means that the actions 
    can be cascaded to the referencing relation or values in the referenced 
    relation can be updated.
    \item Default behaviour is to reject action on the referenced relation that
    causes the violation.
\end{enumerate}

\subsection{Assigning Names to Constraints}
\begin{enumerate}
    \item Syntax: 
    \begin{lstlisting}
    constraint constrName check pred;
    \end{lstlisting}
    \item Helps in dropping constraints. Syntax: 
    \begin{lstlisting}
    alter table tableName drop constraint constrName;
    \end{lstlisting}
    \item Not all database systems support system assigned names for constraints.
\end{enumerate}

\subsection{Integrity Constraint Violation During a Transaction}
\begin{enumerate}
    \item Integrity constraints can be violated in a transaction temporarily 
    after one step, but restored in a later step of the transaction.
    \item To allow for such flexibility, SQL provides a clause 
    \textbf{initially deferred} that can be added to a constraint specification.
    \item Alternatively, a constraint can be specified as \textbf{defferable}, 
    meaning that it is checked immediately by default but can be deferred.
    \item For deferrable constraints, we can specify a list of constraints to 
    be treated as deferred as part of a transaction. Syntax:
    \begin{lstlisting}
    set constraints constrList deferrable;
    \end{lstlisting}
    \item We can workaround the constraints by assigning temporary \textbf{null} 
    values, but that requires a lot of programming effort.
\end{enumerate}

\subsection{Complex Check Conditions and Assertions}
\begin{enumerate}
    \item The predicate in the \textbf{check} clause can be any predicate, and 
    can also contain subqueries.
    \item However, complex check conditions can be costly to test.
    \item An \textbf{assertion} is a predicate expressing a condition that we 
    wish the database always to satisfy.
    \item Syntax: 
    \begin{lstlisting}
    create assertion assertName check pred;
    \end{lstlisting}
\end{enumerate}

\section{SQL Data Types and Schemas}

\subsection{Date and Time Types in SQL}
\begin{enumerate}
    \item 
    \item \textbf{date:} A calendar date containing a (four-digit) year, month 
    and day.
    \item \textbf{time:} The time of day in hours, minutes and seconds. An 
    additional data type \textbf{time($p$)} is used to specify the number of 
    fractional digits (default = 0) for seconds. One can specify timezone 
    information using \textbf{time with timezone}.
    \item \textbf{timestamp:} A combination of \textbf{date} and \textbf{time}. 
    A variant, \textbf{timestamp($p$)} can be used to specify the number of 
    fractional digits (default = 0) for seconds.
    \item To extract individual fields from each data type, we use 
    \textbf{extract} ($f$ \textbf{from} $d$);
    \item SQL defines several functions to get:
    \begin{enumerate}
        \item Current date (\textbf{current\_date}), 
        \item Current time (\textbf{current\_time}), 
        \item Current timestamp with timezone (\textbf{current\_timestamp}),
        \item Local timestamp without timezone (\textbf{local\_timestamp}).
    \end{enumerate}
    \item For comparisons of date and time types, an additional data type 
    called \textbf{interval} is defined by SQL.
    \item An \textbf{interval} is obtained on subtraction of two variables of 
    type \textbf{date} or \textbf{time}. Conversely, adding or subtracting an 
    interval from a \textbf{date} or \textbf{time} gives back a \textbf{date} 
    or \textbf{time} respectively.
\end{enumerate}

\subsection{Type Conversion and Formatting Functions}
\begin{enumerate}
    \item For \textit{explicit} type conversion, the syntax is:
    \begin{lstlisting}
    cast (val as newType);
    \end{lstlisting}
    \item The \textbf{coalesce} function can be used to change how null values 
    are displayed. Syntax:
    \begin{lstlisting}
    coalesce (val_1, val_2, ..., val_n);
    \end{lstlisting}
    It returns the first non-null value from the argument list.
    \item Some database systems allow values of any data type to be shown 
    instead of \textbf{null}, such as the \textbf{decode} function in Oracle.
\end{enumerate}

\subsection{Default Values}
\begin{enumerate}
    \item Syntax (in DDL):
    \begin{lstlisting}
    ...,
    attr domain default val,
    ...,
    \end{lstlisting}
    \item If an attribute has default value, it may not be declared when 
    inserting a tuple into the relation.
\end{enumerate}

\subsection{Large Object Types}
\begin{enumerate}
    \item SQL has two data types to store large data items:
    \begin{enumerate}
        \item \textbf{blob}($s$): This is used to store Binary Large OBjects 
        such as executables, movies, images, etc. upto size $s$.
        \item \textbf{clob}($s$): This is used to store Character Large OBjects 
        such as text files or documents upto size $s$.
    \end{enumerate}
    \item Storing large objects in memory is inefficient and impractical. Hence, 
    a \textit{locator} is stored in the database and retrieved when required to 
    access the data object.
\end{enumerate}

\subsection{User-Defined Types}
\begin{enumerate}
    \item There are two forms of user-defined types: \textbf{structured} and 
    \textbf{distinct}.
    \item Distinct data types are created using the following syntax:
    \begin{lstlisting}
    create type typeName as actualType final;
    \end{lstlisting}
    Note that \textbf{final} is required by the SQL:1999 standard.
\end{enumerate}

\section{Index Definitions in SQL}

\begin{enumerate}
    \item An \textbf{index} is a data structure on an attribute that allows the
    database system to find tuples with a particular value of the attribute 
    efficiently.
    \item Redundant, form part of the physical and not logical schema.
    \item Important for efficient processing of transactions, enforcement of 
    constraints.
    \item Not easy for the database system to determine what indices to maintain.
    Control over index creation and deletion left to programmer.
    \item Syntax for index creation:
    \begin{lstlisting}
        create index indexName on tableName(attrList);
    \end{lstlisting}
    The listed attributes form the search key of the index.
    \item To declare that the search key is a candidate key, add the attribute
    \textbf{unique} to the index definition.
    \item To drop an index:
    \begin{lstlisting}
        drop index indexName;
    \end{lstlisting}
\end{enumerate}

\section{Authorization}

\begin{enumerate}
    \item Includes authorization to 
    \begin{enumerate}
        \item read data
        \item insert new data
        \item update data
        \item delete data
    \end{enumerate}
    Each authorization is called a \textbf{privilege}.
    \item SQL implementation checks whether the user has access to perform the 
    operation before performing the operation.
    \item Ultimate authority resides with database administrators.
\end{enumerate}

\subsection{Granting and Revoking of Privileges}

\begin{enumerate}
    \item SQL standard includes \textbf{select, insert, update} and 
    \textbf{delete} privileges.
    \item To confer authorization:
    \begin{lstlisting}
        grant privList
        on tableName/viewName
        to userList/roleList;
    \end{lstlisting}
    \item For the \textbf{update} privilege, one can optionally specify the authorized
    list of attributes in parentheses after \textbf{update}.
    \item The usernaem \textbf{public} refers to current and future users of
    the database system.
    \item To revoke authorizations:
    \begin{lstlisting}
        revoke privList
        on tableName/viewName
        from userList/roleList;
    \end{lstlisting}
\end{enumerate}

\subsection{Roles}

\begin{enumerate}
    \item Presents a better approach to managing authorizations for various 
    users.
    \item Can grant/revoke authorizations from roles instead of users.
    \item To create roles 
    \begin{lstlisting}
        create role roleName;
    \end{lstlisting}
    \item To grant a role to a user
    \begin{lstlisting}
        grant roleName to userName;
    \end{lstlisting}
\end{enumerate}

\subsection{Authorization on Views}

\begin{enumerate}
    \item When a view is used, the query processor replaces the uses of the view
    with its definition, thereby producing a query on the underlying relation
    itself.
    \item The database system checks authorization on the query of the user.
    \item A user who creates a view receives no additional authorization beyond
    those they already have. \textbf{No additional privileges are granted.}
    \item View creation request is denied if no privileges can be granted on it.
    \item The \textbf{execute} privilege is granted on a function or a procedure.
    The function or procedure runs as if it was exected by the user who created 
    it.
    \item This behaviour is not always required. From SQL:2003 onwards, if the 
    function definition has an extra clause \textbf{sql security invoker}, then 
    it is executed under the privileges of the user who calls the function rather
    than the user who \textit{defined} it.
\end{enumerate}

\subsection{Authorizations on Schema}

\begin{enumerate}
    \item \textit{Only} the schema owner can modify the schema, such as 
    create/delete relations, add/drop attributes or indices.
    \item The \textbf{references} privilege permits a user to declare foreign
    keys when creating relations. The schema can be optionally specified in 
    parentheses after the privilege.
    \item Creating foreign keys with privileges can prevent activity in 
    the referenced relation as the referencing relations may have to be
    updated.
    \item Restricts creation of \textbf{check} constraints in referencing
    relations due to the foreign key privileges.
\end{enumerate}

\subsection{Transfer of Privileges}

\begin{enumerate}
    \item To authorize a user to further grant a privilege to another user/role 
    the clause \textbf{with grant option} is appended to the \textbf{grant} 
    command.
    \item Leads to a notion of an \textbf{authorization graph}, which is a
    directed graph.
    \item The nodes of this graph are the users. If there is an edge $U_i 
    \rightarrow U_j$ in this graph, then user $U_i$ grants the privilege to 
    $U_j$.
    \item A user has an authorization \textit{iff} there is a path from the root
    of the authorization graph (representing the DBA) to the node representing
    the user.
\end{enumerate}

\subsection{Revoking of Privileges}

\begin{enumerate}
    \item Revocation of a privilege from a user/role may cause \textit{cascading}
    revocation. Default behaviour in database systems.
    \item To prevent this, append the revoke statement with the \textbf{revoke}
    clause. Error is raised and revocations are not performed if there are
    cascading revocations.
    \item To revoke the grant option:
    \begin{lstlisting}
        revoke grant option for privList on tableName from userName;
    \end{lstlisting}
    \item Cascading revocation is not always appropriate. To prevent ill effects
    of the same, SQL permits granting of privileges by roles to other roles.
    \item SQL has a notion of current role in a session. Usually set to null by 
    default. Change using
    \begin{lstlisting}
        set role roleName;
    \end{lstlisting}
    The statement executes if the role has been granted to the user.
    \item To grant a privilege from the current role, append the \textbf{grant}
    statement with \textbf{granted by current\_role} clause.
\end{enumerate}

\subsection{Row-Level Authorization}

\begin{enumerate}
    \item Fine-grained authorization. Prevents users from seeing irrelevant 
    data.
    \item Oracle's \textbf{Virtual Private Database} (VPD) supports this 
    mechanism.
    \item Database system adds predicates to \textbf{where} clauses to 
    enforce the authorization.
    \item A pitfall of this mechanism is that it can alter the meanings of
    issued queries.
\end{enumerate}

\end{document}