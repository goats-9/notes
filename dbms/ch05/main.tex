\documentclass[journal,12pt,twocolumn]{IEEEtran}
\usepackage{setspace}
\usepackage{gensymb}
\usepackage{xcolor}
\usepackage{caption}
\singlespacing
\usepackage{siunitx}
\usepackage[cmex10]{amsmath}
\usepackage{mathtools}
\usepackage{hyperref}
\usepackage{amsthm}
\usepackage{mathrsfs}
\usepackage{txfonts}
\usepackage{stfloats}
\usepackage{cite}
\usepackage{cases}
\usepackage{subfig}
\usepackage{longtable}
\usepackage{multirow}
\usepackage{enumitem}
\usepackage{mathtools}
\usepackage{listings}
\usepackage{tikz}
\usetikzlibrary{shapes,arrows,positioning}
\usepackage{circuitikz}
\usepackage{tabularx}
\let\vec\mathbf
\DeclareMathOperator*{\Res}{Res}
\renewcommand\thesection{\arabic{section}}
\renewcommand\thesubsection{\thesection.\arabic{subsection}}
\renewcommand\thesubsubsection{\thesubsection.\arabic{subsubsection}}

\renewcommand\thesectiondis{\arabic{section}}
\renewcommand\thesubsectiondis{\thesectiondis.\arabic{subsection}}
\renewcommand\thesubsubsectiondis{\thesubsectiondis.\arabic{subsubsection}}
\hyphenation{op-tical net-works semi-conduc-tor}

\lstset{
language=SQL,
frame=single, 
breaklines=true,
columns=fullflexible,
showstringspaces=false
}
\begin{document}
\theoremstyle{definition}
\newtheorem{theorem}{Theorem}[section]
\newtheorem{problem}{Problem}
\newtheorem{proposition}{Proposition}[section]
\newtheorem{lemma}{Lemma}[section]
\newtheorem{corollary}[theorem]{Corollary}
\newtheorem{example}{Example}[section]
\newtheorem{definition}{Definition}[section]
\newcommand{\BEQA}{\begin{eqnarray}}
\newcommand{\EEQA}{\end{eqnarray}}
\newcommand{\define}{\stackrel{\triangle}{=}}
\newcommand{\myvec}[1]{\ensuremath{\begin{pmatrix}#1\end{pmatrix}}}
\newcommand{\mydet}[1]{\ensuremath{\begin{vmatrix}#1\end{vmatrix}}}
\bibliographystyle{IEEEtran}
\providecommand{\nCr}[2]{\,^{#1}C_{#2}} % nCr
\providecommand{\nPr}[2]{\,^{#1}P_{#2}} % nPr
\providecommand{\mbf}{\mathbf}
\providecommand{\pr}[1]{\ensuremath{\Pr\left(#1\right)}}
\providecommand{\qfunc}[1]{\ensuremath{Q\left(#1\right)}}
\providecommand{\sbrak}[1]{\ensuremath{{}\left[#1\right]}}
\providecommand{\lsbrak}[1]{\ensuremath{{}\left[#1\right.}}
\providecommand{\rsbrak}[1]{\ensuremath{{}\left.#1\right]}}
\providecommand{\brak}[1]{\ensuremath{\left(#1\right)}}
\providecommand{\lbrak}[1]{\ensuremath{\left(#1\right.}}
\providecommand{\rbrak}[1]{\ensuremath{\left.#1\right)}}
\providecommand{\cbrak}[1]{\ensuremath{\left\{#1\right\}}}
\providecommand{\lcbrak}[1]{\ensuremath{\left\{#1\right.}}
\providecommand{\rcbrak}[1]{\ensuremath{\left.#1\right\}}}
\theoremstyle{remark}
\newtheorem{rem}{Remark}
\newcommand{\sgn}{\mathop{\mathrm{sgn}}}
\newcommand{\rect}{\mathop{\mathrm{rect}}}
\newcommand{\sinc}{\mathop{\mathrm{sinc}}}
\providecommand{\abs}[1]{\left\vert#1\right\vert}
\providecommand{\res}[1]{\Res\displaylimits_{#1}} 
\providecommand{\norm}[1]{\lVert#1\rVert}
\providecommand{\mtx}[1]{\mathbf{#1}}
\providecommand{\mean}[1]{E\left[ #1 \right]}
\providecommand{\fourier}{\overset{\mathcal{F}}{ \rightleftharpoons}}
\providecommand{\ztrans}{\overset{\mathcal{Z}}{ \rightleftharpoons}}
\providecommand{\system}[1]{\overset{\mathcal{#1}}{ \longleftrightarrow}}
\newcommand{\solution}{\noindent \textbf{Solution: }}
\providecommand{\dec}[2]{\ensuremath{\overset{#1}{\underset{#2}{\gtrless}}}}
\numberwithin{equation}{section}
\makeatletter
\@addtoreset{figure}{problem}
\makeatother
\let\StandardTheFigure\thefigure
\renewcommand{\thefigure}{\theproblem}
\def\putbox#1#2#3{\makebox[0in][l]{\makebox[#1][l]{}\raisebox{\baselineskip}[0in][0in]{\raisebox{#2}[0in][0in]{#3}}}}
     \def\rightbox#1{\makebox[0in][r]{#1}}
     \def\centbox#1{\makebox[0in]{#1}}
     \def\topbox#1{\raisebox{-\baselineskip}[0in][0in]{#1}}
     \def\midbox#1{\raisebox{-0.5\baselineskip}[0in][0in]{#1}}
\vspace{3cm}
\title{Chapter 5: Advanced SQL}
\maketitle
\tableofcontents
\renewcommand{\thefigure}{\theenumi}
\renewcommand{\thetable}{\theenumi}
\bigskip

\section{Accessing SQL from a Programming Language}

Two ways of doing so:
\begin{enumerate}
     \item \textbf{Dynamic SQL:}
     \begin{enumerate}
          \item Program can construct an SQL query as a string at runtime,
          submit the query and receive results in the form of tuples.
          \item Examples are JDBC, ODBC, etc.
     \end{enumerate}
     \item \textbf{Embedded SQL:}
     \begin{enumerate}
          \item SQL statements identified at compile time with a preprocessor.
          \item Preprocessor translates requests in embedded SQL into function
          calls.
          \item At runtime, function calls connect to the database using API
          that provides dynamic SQL facilities.
     \end{enumerate}
\end{enumerate}

A major challenge of accessing SQL is the difference in how SQL and the general
purpose programming language treat data. SQL considers tuples in a relation
while the programing language uses variables. A mechanism is needed to return
values in a form that the programming language can handle.

\subsection{JDBC}

\begin{enumerate}
     \item Defines an \textbf{application program interface} (API) that Java
     programs can use to connect to database servers.
     \item Java program must import \texttt{java.sql.*} for use.
\end{enumerate}

\subsubsection{Connecting to the Database}\hfill

The \texttt{DriverManager.getConnection()} function is used. It takes three 
parameters.
\begin{enumerate}
     \item String in \texttt{protocol@url:port:db} format. For example,
     \texttt{jdbc:oracle:thin:}. JDBC specifies an API and \textit{not} a 
     communication protocol. So, one must specify a protocol each supported by 
     the database and the JDBC driver.
     \item String which is the database user identifier.
     \item String which is a password (security risk if password specified in code).
\end{enumerate}
The significance of drivers are listed below.
\begin{enumerate}
     \item Database products that support JDBC provide a JDBC driver that must
     be dynamically loaded to access the database from Java.
     \item The \texttt{getConnection()} method will locate the driver if
     downloaded and placed in the classpath.
     \item Drivers translate product-independent JDBC calls into product-specific
     calls needed by the database management system
     \item Protocol used to exchange information depends on the driver used, and
     is not JDBC standard. Some drivers can support more than one protocol.
\end{enumerate}

\subsubsection{Shipping SQL Statements to the Database System}\hfill

\begin{enumerate}
     \item A \texttt{Statement} object is used. It allows Java to invoke
     methods to ship an SQL statement as an argument for execution by the
     databse system.
     \item Execute the statement using \texttt{executeQuery()} or 
     \texttt{executeUpdate()}, depending on whether the SQL statement is a
     query or a nonquery statement.
     \item \texttt{executeUpdate()} returns the number of tuples updated,
     inserted, or deleted. It is zero for DDL statements.
\end{enumerate}

\subsubsection{Exceptions and Resource Management}\hfill

\begin{enumerate}
     \item Executing a SQL method can result in an \texttt{SQLexception} being
     thrown. This is an SQL-specific exception as compared to \texttt{Exception}
     class for any general exception.
     \item Opening a JDBC connection, creating JDBC objects consumes system 
     resources. Programmers must ensure that resources are freed after use, to 
     prevent resource exhaustion or access to system till timeout.
     \item Explicitly freeing resources may not work if an exception is thrown
     earlier. 
     \item Preferred approach is to use \textit{try-with-resources} block.
     Resources are automatically closed at the end of this block. Resources are
     declared in parentheses and statements follow in curly braces.
\end{enumerate}

\subsubsection{Retrieving the Result of a Query}\hfill

\begin{enumerate}
     \item The results of an SQL query are returned as a \texttt{ResultSet} 
     object.
     \item The \texttt{next()} method returns a Boolean testing whether
     there is at least one unfetched tuple in the \texttt{ResultSet}.
     \item Retrieving attributes can also be done using \texttt{get} methods
     such as \texttt{getString()} or \texttt{getFloat()}. The name of the 
     attribute or the column number (starting from 1) is passed as the 
     argument.
\end{enumerate}

\begin{lstlisting}[language=java,caption={Example Usage of JDBC}]
public static void JDBCexample(String userid, String passwd)
{
     try (
          Connection conn = DriverManager.getConnection(
               "jdbc:oracle:thin:@db.yale.edu:1521:univdb",
               userid, passwd);
          Statement stmt = conn.createStatement();
     ){
     try {
          stmt.executeUpdate(
               "insert into instructor values(`77987`,`Kim`,`Physics`,98000)");
     }
     catch (SQLException sqle) {
          System.out.println("Could not insert tuple. " + sqle);
     }
     ResultSet rset = stmt.executeQuery(
          "select dept name, avg (salary) "+
          " from instructor "+
          " group by dept name");
     while (rset.next()) {
          System.out.println(rset.getString("dept name") + " " +
               rset.getFloat(2));
     }
     }
     catch (Exception sqle)
     {
          System.out.println("Exception : " + sqle);
     }
} 
\end{lstlisting}

\subsubsection{Prepared Statements}

\begin{enumerate}
     \item Statement in which some values are replaced by ``?'', meaning that
     actual values will be provided later.
     \item Database system compiles query when it is prepared, and reuse the
     compiled query by applying values as parameters.
     \item In Java, prepared statements are created using the 
     \texttt{prepareStatement} method of the \texttt{Connection} class. It
     returns an object of the \texttt{PreparedStatement} class.
     \item Use \texttt{executeQuery()} or \texttt{executeUpdate()} 
     method of the \texttt{PreparedStatement} class to execute.
     \item Values can be passed using \texttt{setString()} and other such 
     methods.
     \item Allows for more efficient execution in cases where the same query
     is to be run with different values.
     \item Prevents \textbf{SQL injection} attacks which occur when
     concatenated statements are used. In these cases, special characters are
     not escaped and one can break the query, then subsequently execute other
     queries with privileges of the user specified in the connection.
     \item Prepared statements escape characters in the input string, 
     preventing SQL injection attacks.
\end{enumerate}

\subsubsection{Callable Statements}

\begin{enumerate}
     \item \texttt{CallableStatement} interface allows for invocation of SQL
     stored functions or procedures.
     \item \texttt{prepareCall()} method works the same way as 
     \texttt{prepareStatement()} method and returns a \texttt{CallableStatement} 
     object.
     \item Data types of return values must be registed using the
     \texttt{registerOutParameter()} class, and retrieved using get methods.
\end{enumerate}

\subsubsection{Metadata Features}
\begin{enumerate}
     \item \texttt{getMetaData()} method of the \texttt{ResultSet} class returns
     \texttt{ResultSetMetaData} object that contains metadata about the
     \texttt{ResultSet} object.
     \item Various methods in the \texttt{ResultSetMetaData} interface:
     \begin{enumerate}
          \item \texttt{getColumnCount()}: Returns the number of columns in 
          the result set.
          \item \texttt{getColumnName()}: Takes the column number as an argument,
          returns the corresponding name.
          \item \texttt{getColumnTypeName()}: Takes the column number as an argument,
          returns the corresponding name of the type.
     \end{enumerate}
     \item \texttt{DatabaseMetaData} interface contains metadata about the database.
     \item \texttt{Connection} interface has \texttt{getMetaData()} method that 
     returns a \texttt{DatabaseMetaData} object.
     \item Various methods in the \texttt{DatabaseMetaData} interface:
     \begin{enumerate}
          \item \texttt{getColumns()}: Takes four arguments - a catalog name (can be
          ignored if null is used), schema name pattern, table name pattern and a 
          column name pattern. SQL string matching sequences may be used.

          A \texttt{ResultSet} is returned which containins columns of tables 
          of schemas satisfying the patterns. The columns in this result set
          contain the name of the catalog, schema, table and column, the type
          of the column and so on.
          \item \texttt{getTables()}: Returns a list of all tables in the
          database. First three arguments are the same as before, fourth 
          restricts types of tables returned. If it is \texttt{null}, then all 
          including system internal tables are returned.
          \item \texttt{getPrimaryKeys()} method for getting primary keys,
          \texttt{getCrossReference()} for foreign keys, etc.
     \end{enumerate}
\end{enumerate}

\subsubsection{Other Features}

\begin{enumerate}
     \item JDBC provides \textbf{updatable result sets}, which are created from a query
     that returns a relation. Updates to such a result set are reflected in the database.
     \item Auto-commiting each query is controlled using \texttt{Connection.setAutoCommit()} 
     and passing a Boolean to it.
     \item Explicit commit/rollback can be done when autocommit is off using 
     \texttt{Connection.commit()} and \texttt{Connection.rollback()}.
     \item \texttt{getBlob()} and \texttt{getClob()} methods return pointers
     to the large object. Fetch the data using \texttt{getBytes()} and
     \texttt{getSubString()} methods.
     \item Store large objects using \texttt{setBlob(int idx, InputStream stream)}
     for \textbf{blob} objects and \texttt{setClob()} for \textbf{clob} 
     objects.
     \item \textit{Row set} feature allows results to be collected and shipped topbox
     other applications. They can be scanned in either direction and modified.
\end{enumerate}

\subsection{ODBC}

\begin{enumerate}
     \item The \textbf{Open Database Connectivity Standard} (ODBC) defines an 
     API that applications use to open connections with databses, send queries,
     and get resutls.
     \item To connect, use the \texttt{SQLConnect} function. Parameters passed
     are connection handle, server, userid, password, etc.
     \item \texttt{SQL\_NTS} indicates previous argument is a null-terminated 
     string. \texttt{SQL\_SUCCESS} indicates successful operation.
     \item \texttt{SQLExecDirect()} is used to execute SQL statements.
     \item \texttt{SQLFetch()} fetches the next row in the result.
     \item \texttt{SQLBindCol()} stores attribute values in C variables.
     \item \texttt{SQLSetConnectOption()} is used to change autocommit option.
     \item \texttt{SQLTransact()} is used for commit or rollback explicitly.
     \item \textit{Conformance levels} in ODBC:
     \begin{enumerate}
          \item \textbf{Level 1:} Support for fetching information about the 
          catalog, i.e., relations and types of attributes in them.
          \item \textbf{Level 2:} Ability to send and retrieve arrays of
          parameter values and more detailed catalog information.
     \end{enumerate}
     \item SQL standard defines a \textbf{call level interface} (CLI) that is 
     similar to ODBC.
\end{enumerate}

\subsection{Embedded SQL}

\begin{enumerate}
     \item Language in which SQL is embedded is called the \textit{host}
     langugage. Permitted SQL structures constitute \textit{embedded} SQL.
     \item Special preprocessors are needed to process embedded SQL prior to 
     compilation. This is the main difference from ODBC/JDBC.
     \item For identifying embedded SQL, we use EXEC SQL $<$SQL statement$>$;
     \item Variables from the host program can be used in embedded SQL but 
     must be prepended using a colon.
     \item To iterate over the results of a quer, we must declare a 
     \textit{cursor} variable, which is then opened and used to fetch tuples
     in the host program. They can also be used to perform updates.
     \item \textit{Advantages}: SQL-related errors caught in preprocessing.
     Easier to comprehend SQL statements in this form.
     \item \textit{Disadvantages}: Preprocessor creates new host language code,
     difficult to debug. Preprocessor constructs may clash with host language
     syntax in later versions.
     \item Most current systems use \textit{dynamic} SQL instead, except
     Microsoft Language Integrated Query (LINQ) faciility, which extends host 
     language support to include embedded queries into host language. 
\end{enumerate}

\section{Functions and Procedures}

\begin{enumerate}
     \item Store ``business logic'' in the database and executed from SQL 
     statements.
     \item \textit{Advantages}: Allows multiple applications to access common 
     procedures. Changes need to be made at one point only.
\end{enumerate}

\subsection{Declaring and Invoking SQL Functions and Procedures}
\begin{enumerate}
     \item Functions that reutrn tables are called \textbf{table functions}. Syntax:
     \begin{lstlisting}
create function funcName(params)
returns table (
     schema
)
return table (
     query
);
     \end{lstlisting}
     \item Table functions can be thought of as \textbf{parameterized views}.
     \item Refer to parameters by prefixing them with the function name.
     \item Can be used in queries as shown:
     \begin{lstlisting}
select *
from table(func(args));
     \end{lstlisting}
     \item SQL supports procedures. Syntax:
     \begin{lstlisting}
create procedure procName((in|out) param1 type1, (in|out) param2 type2, ...)
begin
     <statements>
end
     \end{lstlisting}
     \item Keywords \textbf{in} and \textbf{out} indicate parameters that have
     assigned values and parameters whose values are set by the procedure
     respectively.
     \item Invoke procedures using \textbf{call}:
     \begin{lstlisting}
declare var type default val;
call procName(params);
     \end{lstlisting}
     \item Permitted to have more than one procedure with the same name, as long 
     as their argument lists are different.
\end{enumerate}

\subsection{Language Constructs for Procedures and Functions}
\begin{enumerate}
     \item \textbf{Persistent Storage Module} (PSM) is the part of the SQL 
     standard that deals with constructs that are seen in general-purpose 
     programming languages.
     \item Variables are declared using \textbf{declare}.
     \item Assignments are performed using \textbf{set}.
     \item Compound statements consist of many SQL statements that are
     enclosed in \textbf{begin\ldots end}.
     \item If enclosed in \textbf{begin atomic\ldots end}, the statements are
     considered to be part of a transaction.
     \item Supports \textbf{while}, \textbf{for} and \textbf{repeat} constructs.
     \item In \textbf{for} loops, we may use \textbf{leave} to break out of the
     loop and \textbf{iterate} starts the next iteration of the loop.
     \item Conditional statements supported by SQL include 
     \textbf{if-then-elseif-else} statements.
\end{enumerate}

\end{document}