\documentclass[journal,12pt,twocolumn]{IEEEtran}
\usepackage{setspace}
\usepackage{gensymb}
\usepackage{xcolor}
\usepackage{caption}
\singlespacing
\usepackage{siunitx}
\usepackage[cmex10]{amsmath}
\usepackage{mathtools}
\usepackage{hyperref}
\usepackage{amsthm}
\usepackage{mathrsfs}
\usepackage{txfonts}
\usepackage{stfloats}
\usepackage{cite}
\usepackage{cases}
\usepackage{subfig}
\usepackage{longtable}
\usepackage{multirow}
\usepackage{enumitem}
\usepackage{mathtools}
\usepackage{listings}
\usepackage{tikz}
\usetikzlibrary{shapes,arrows,positioning}
\usepackage{circuitikz}
\usepackage{tabularx}
\let\vec\mathbf
\DeclareMathOperator*{\Res}{Res}
\renewcommand\thesection{\arabic{section}}
\renewcommand\thesubsection{\thesection.\arabic{subsection}}
\renewcommand\thesubsubsection{\thesubsection.\arabic{subsubsection}}

\renewcommand\thesectiondis{\arabic{section}}
\renewcommand\thesubsectiondis{\thesectiondis.\arabic{subsection}}
\renewcommand\thesubsubsectiondis{\thesubsectiondis.\arabic{subsubsection}}
\hyphenation{op-tical net-works semi-conduc-tor}

\lstset{
language=sql,
frame=single, 
breaklines=true,
columns=fullflexible
}
\begin{document}
\theoremstyle{definition}
\newtheorem{theorem}{Theorem}[section]
\newtheorem{problem}{Problem}
\newtheorem{proposition}{Proposition}[section]
\newtheorem{lemma}{Lemma}[section]
\newtheorem{corollary}[theorem]{Corollary}
\newtheorem{example}{Example}[section]
\newtheorem{definition}{Definition}[section]
\newcommand{\BEQA}{\begin{eqnarray}}
\newcommand{\EEQA}{\end{eqnarray}}
\newcommand{\define}{\stackrel{\triangle}{=}}
\newcommand{\myvec}[1]{\ensuremath{\begin{pmatrix}#1\end{pmatrix}}}
\newcommand{\mydet}[1]{\ensuremath{\begin{vmatrix}#1\end{vmatrix}}}
\bibliographystyle{IEEEtran}
\providecommand{\nCr}[2]{\,^{#1}C_{#2}} % nCr
\providecommand{\nPr}[2]{\,^{#1}P_{#2}} % nPr
\providecommand{\mbf}{\mathbf}
\providecommand{\pr}[1]{\ensuremath{\Pr\left(#1\right)}}
\providecommand{\qfunc}[1]{\ensuremath{Q\left(#1\right)}}
\providecommand{\sbrak}[1]{\ensuremath{{}\left[#1\right]}}
\providecommand{\lsbrak}[1]{\ensuremath{{}\left[#1\right.}}
\providecommand{\rsbrak}[1]{\ensuremath{{}\left.#1\right]}}
\providecommand{\brak}[1]{\ensuremath{\left(#1\right)}}
\providecommand{\lbrak}[1]{\ensuremath{\left(#1\right.}}
\providecommand{\rbrak}[1]{\ensuremath{\left.#1\right)}}
\providecommand{\cbrak}[1]{\ensuremath{\left\{#1\right\}}}
\providecommand{\lcbrak}[1]{\ensuremath{\left\{#1\right.}}
\providecommand{\rcbrak}[1]{\ensuremath{\left.#1\right\}}}
\theoremstyle{remark}
\newtheorem{rem}{Remark}
\newcommand{\sgn}{\mathop{\mathrm{sgn}}}
\newcommand{\rect}{\mathop{\mathrm{rect}}}
\newcommand{\sinc}{\mathop{\mathrm{sinc}}}
\providecommand{\abs}[1]{\left\vert#1\right\vert}
\providecommand{\res}[1]{\Res\displaylimits_{#1}} 
\providecommand{\norm}[1]{\lVert#1\rVert}
\providecommand{\mtx}[1]{\mathbf{#1}}
\providecommand{\mean}[1]{E\left[ #1 \right]}
\providecommand{\fourier}{\overset{\mathcal{F}}{ \rightleftharpoons}}
\providecommand{\ztrans}{\overset{\mathcal{Z}}{ \rightleftharpoons}}
\providecommand{\system}[1]{\overset{\mathcal{#1}}{ \longleftrightarrow}}
\newcommand{\solution}{\noindent \textbf{Solution: }}
\providecommand{\dec}[2]{\ensuremath{\overset{#1}{\underset{#2}{\gtrless}}}}
\numberwithin{equation}{section}
\makeatletter
\@addtoreset{figure}{problem}
\makeatother
\let\StandardTheFigure\thefigure
\renewcommand{\thefigure}{\theproblem}
\def\putbox#1#2#3{\makebox[0in][l]{\makebox[#1][l]{}\raisebox{\baselineskip}[0in][0in]{\raisebox{#2}[0in][0in]{#3}}}}
     \def\rightbox#1{\makebox[0in][r]{#1}}
     \def\centbox#1{\makebox[0in]{#1}}
     \def\topbox#1{\raisebox{-\baselineskip}[0in][0in]{#1}}
     \def\midbox#1{\raisebox{-0.5\baselineskip}[0in][0in]{#1}}
\vspace{3cm}
\title{Chapter 1: Introduction}
\maketitle
\tableofcontents
\renewcommand{\thefigure}{\theenumi}
\renewcommand{\thetable}{\theenumi}
\bigskip

\section{Purpose of Database Systems}

Conventional operating systems store records in files, forming a 
\textit{file-processing system}. The advantages of a database system over a 
file system are:

\begin{enumerate}
    \item \textbf{Data redundancy and inconsistency:}
    \begin{enumerate}
        \item Various files created over time have different structures.
        \item Application programs may be written in several programming 
        languages.
        \item Same information may be duplicated in several files, records may
         not match. This is called \textit{data inconsistency}.
    \end{enumerate}

    \item \textbf{Difficulty in Accessing Data:}
    \begin{enumerate}
        \item With a file system, a query must either be \textit{manually}
         processed, or an application program must be written.
        \item File systems do not allow data retrieval and querying in an 
        efficient manner.
    \end{enumerate}

    \item \textbf{Data isolation:}
    \begin{enumerate}
        \item Data scattered in various files and formats.
        \item Difficult to write new applications to retireve data.
    \end{enumerate}

    \item \textbf{Integrity problems:}
    \begin{enumerate}
        \item Data in a database must satisfy some \textit{consistency 
        constraints}.
        \item Adding new constraints to application programs is difficult.
    \end{enumerate}

    \item \textbf{Atomicity problems:}
    \begin{enumerate}
        \item For database consistency, an operation must either occur entirely
        or not at all.
        \item Difficult to ensure atomicity in a file system.
    \end{enumerate}

    \item \textbf{Concurrent-access anomalies:}
    \begin{enumerate}
        \item For performance and faster response, systems allow multiple users
        to update the data simultaneously.
        \item Concurrent updates may interact and result in inconsistent data.
        \item To guard against this, the system must maintain some form of 
        supervision. Difficult in file systems.
    \end{enumerate}

    \item \textbf{Security problems:}
    \begin{enumerate}
        \item Users should have restricted access to prevent tampering of data.
        \item Difficult to enforce restrictions in application programs.
    \end{enumerate}
\end{enumerate}

\section{View of Data}

\subsection{Data Models}

A \textbf{data model} is a collection of conceptual tools for describing data, 
data relationships/semantics and consistency constraints. They are classified as:

\begin{enumerate}
    \item \textbf{Relational Model:}
    \begin{enumerate}
        \item Collection of tables (called \textit{relations}) used.
        \item Each table has multiple columns (each with their unique name), 
        correspinding to the attributes of the records.
        \item Each record has a fixed number of attributes.
        \item Most widely used data model.
    \end{enumerate}

    \item \textbf{Entity-Relationship (E-R) Model:}
    \begin{enumerate}
        \item Uses a collection of basic objects (called \textit{entities}) and
        \textit{relationships} among those objects.
        \item Most widely used in database design.
    \end{enumerate}

    \item \textbf{Semi-structured Data Model:}
    \begin{enumerate}
        \item Permits the specification of data where individual items can have
        different sets of attributes.
        \item Examples are \textit{JSON} and \textit{XML} (eXtensible Markup
        Language).
    \end{enumerate}

    \item \textbf{Object-Based Data Model:}
    \begin{enumerate}
        \item Well integrated into relational databases.
        \item Can store objects and procedures to execute in the database system.
        \item Can be seen as extending the relational model with notions of 
        encapsulation, methods, and object identity.
    \end{enumerate}
\end{enumerate}

\subsection{Data Abstraction}

The complexity of the database system is hidden from users through several 
levels of \textit{data abstraction} as follows:

\begin{enumerate}
    \item \textbf{Physical Level:}
    \begin{enumerate}
        \item Describes \textit{how} the data are actually stored.
        \item Descirbes complex low-level data structures in detail.
        \item Records can be described as a block of consecutive bytes. Hidden 
        by compilers and database systems from programmers.
        \item \textit{Index} data structure is used to support efficient 
        retrieval of records, part of the physical level.
    \end{enumerate}

    \item \textbf{Logical Level:}
    \begin{enumerate}
        \item Describes \textit{what} data are stored in the database, and 
        their relationships.
        \item Describes the database in terms of small number of relatively 
        simple structures.
        \item The user of the logical level does not need to be aware of the 
        complexity of underlying physical-level structures. This is called 
        \textit{physical data independence}.
        \item Each record is described by a type definition. The relationships 
        between these types is also defined at this level.
        \item Used by database administrators and programmers.
    \end{enumerate}

    \item \textbf{View Level:}
    \begin{enumerate}
        \item Highest level of abstraction. Describes only part of the entire 
        database.
        \item Simplifies user interaction with the database system by showing 
        part of the database required. Hides undelying complexities from end users.
        \item Also provides a security mechanism to prevent unrestricted access.
    \end{enumerate}
\end{enumerate}

\subsection{Instances and Schemas}

\begin{enumerate}
    \item \textbf{Instance:} Collection of information stored in the databse 
    at a particular moment.
    \item \textbf{Schema:} The overall design of the database.
    \begin{enumerate}
        \item \textit{Physical schema} describes database design at a physical 
        level.
        \item \textit{Logical schema} describes database design at a logical 
        level. Applications are constructed using the logical schema.
        \item \textit{Subschemas} at the view level describe different views of 
        the database.
    \end{enumerate}
    \item Schemas if poorly created can have problems such as duplication and 
    redundancy.
\end{enumerate}

\section{Database Languages}

\subsection{Data-Definition Language (DDL)}

\begin{enumerate}
    \item Used to specify the database schema and additional properties of the 
    data.
    \item \textit{Data storage and definition} language can be used to specify 
    the storage structure and access methods of the data.
    \item Database systems implement constraints that can be tested with 
    minimal overhead:
    \begin{enumerate}
        \item \textbf{Domain Constraints:} Constrains the values an attribute 
        can take, by associating a domain with every atribute.
        \item \textbf{Referential integrity:} Ensure that a value of a certain 
        attribute(s) appearing in one relation also appears in another relation.
        \item \textbf{Authorization:} To differentiate the users based on the 
        type of access. Database systems support \textit{read, insert, update, 
        delete} authorizations for this purpose.
    \end{enumerate}
    \item DDL outputs are placed in the \textbf{data dictionary}, which 
    contains \textbf{metadata} (data about data). The data dictionary is a 
    special table that can only be accessed by the database system and not by 
    any user.
    \item Format in SQL
 
    \begin{lstlisting}
    create table t
    ( 
        <attr1> <domain1> <constraint1>,
        <attr2> <domain2> <constraint2>,
        ... ,
        <addn-constraint1>,
        <addn-constraint2>,
        ...,
    );
    \end{lstlisting}

\end{enumerate}

\subsection{Data Manipulation Language (DML)}

Enables users to access or manipulate data organized by the data model. Various 
types of accesses are:
\begin{enumerate}
    \item Retrieval of information
    \item Insertion of new information
    \item Deletion of information
    \item Modification of information
\end{enumerate}

There are two types of DMLs:

\begin{table}[!ht]
    \centering
    \begin{tabularx}{\columnwidth}{|X|X|}
    \hline
    \textbf{Procedural DMLs} & \textbf{Declarative or Non-procedural DMLs} \\
    \hline
    Specify \textit{what} data is needed and \textit{how} to get 
    those data. & Specify \textit{what} data is needed 
    \textit{without} specifying how to get those data. \\
    \hline
    Tougher to learn. & Easier to learn. \\
    \hline
    \end{tabularx}
    \label{tab:diff}
\end{table}

\textbf{Query:} Statement requesting the retrieval of information. The part of 
a DML that deals with information retrieval is called a \textit{query language}.

In SQL,
\begin{enumerate}
    \item The SQL query language is nonprocedural.
    \item A query takes inputs of several (possibly one) table(s) and outputs a
    single table.
\end{enumerate}

\subsection{Database Access from Application Programs}
\begin{enumerate}
    \item Nonprocedural query languages are not Turing complete.
    \item Two ways to integrate them using a general-purpose programming language:
    \begin{enumerate}
        \item \textit{Embedded SQL:} SQL statements embedded in a \textit{host} 
        programming language.
        \item \textit{Application Program Interface (API):} Set of procedures 
        used to send queries to the database from the \textit{host} programming
        language.
        \item Examples are Open Database Connectivity (ODBC) standard for 
        languages like C, and Java Database Connectivity (JDBC) standard for 
        Java.
    \end{enumerate}
\end{enumerate}

\section{Database Design}

\begin{enumerate}
    \item Mainly involves design of the database schema.
    \item Initial phase: characterize data needs of the users; interact with 
    domain experts and users for this. Leads to a specification of user 
    requirements.
    \item \textit{Conceptual-design phase:} Requirements translated into a 
    conceptual database schema. Review to prevent conflicts between requirements.
    Focus on describing data and their relationships, using the E-R model and 
    various algorithms to quantify attributes.
    \item \textit{Specification of functional requirements:} Users describe the 
    kinds of operations that should be supported. Design reviewed for such support.
    \item \textit{Logical-design phase:} Map high-level conceptual schema onto 
    the implementation data model of the database system.
    \item \textit{Physical-design phase:} Specify physical features of the database.
\end{enumerate}

\section{Database Engine}

A database system contains various components, described ahead.

\subsection{Storage Manager}

Provides interface between the low-level data and the application 
programs/queries submitted to the system. Converts DML statements into low-level
file-system commands. Responsible for storing, retrieving and updating data in
the database.

Components of the storage manager are as follows:
\begin{enumerate}
    \item \textbf{Authorization and integrity manager:} Checks whether the 
    integrity constraints are satisfied and checks user authority.
    \item \textbf{Transaction manager:} Ensures the database remains in a 
    consistent state despite system failures, and that concurrent transactions 
    occured without conflicts.
    \item \textbf{File manager:} Manages allocation of space on disk and data 
    structures used to represent information stored on disk.
    \item \textbf{Buffer manager:} Fetches data from disk storage, and decides 
    what data to cache in main memory. Critical since it enables databases to 
    handle data sizes much larger than the size of main memory.
\end{enumerate}

Data structures implemented by the storage manager are:
\begin{enumerate}
    \item \textbf{Data files:} Store the database itself
    \item \textbf{Data dictionary:} Stores metadata of the database.
    \item \textbf{Indices:} Provide fast access to data items.
\end{enumerate}

\subsection{Query Processor}

Components of the query processor are as follows:
\begin{enumerate}
    \item \textbf{DDL interpreter:} Interprets DDL statements and records the 
    definitions in the data dictionary.
    \item \textbf{DML compiler:} Translates DML statements into an evalutaion 
    plan of low-level instructions that the query-evaluation engine understands.
    Also performs \textit{query optimization i.e.,} picks the lowest cost 
    evaluation plan.
    \item \textbf{Query evaluation engine:} Executes low-level instructions 
    generated by the DML compiler.
\end{enumerate}

\subsection{Transaction Management}

\textbf{Atomicity:} Single transaction on a database must happen entirely or 
not at all.
\textbf{Consistency:} Database must be correct at any point of time.
\textbf{Durability:} After successful transaction, the new records must persist,
despite system failure.
\textbf{Transaction:} A collection of operations that performs a single logical
function in a database application.

\begin{enumerate}
    \item Each transaction is a unit of atomicity and consistency.
    \item State of consistency must be maintained before and after the 
    transaction.
\end{enumerate}

\textbf{Transaction manager} consists of:
\begin{enumerate}
    \item \textbf{Recovery manager:} Ensures atomicity and durability of the 
    database. Must perform \textit{failure recovery} to detect failures and 
    restore the database to a consistent state.
    \item \textbf{Concurrency-control manager:} Control interaction among 
    concurrent transactions to maintain the consistency of the database.
\end{enumerate}

\section{Database and Application Architecture}

To scale up to larger data sizes and higher processing speeds, we have 
\textit{parallel databases} (run on a cluster consisting of multiple machines)
and \textit{distributed databases} (data stored/queries processed across 
multiple geographically seaparated machines).

Database applications can be partitioned into two or three parts:
\begin{enumerate}
    \item \textit{Two-tier architecture:} Application resides at the client 
    machine and queries the server machine.
    \item \textit{Three-tier architecture:} Used by modern applications. 
    Frontend communicates with the \textit{application server}, which has 
    \textit{business logic} embedded in it (as opposed to deploying it on 
    various clients). Provides better security and performance.
\end{enumerate}

\section{Database Users and Administrators}

There are four types of database users based on the way they interact with the 
system:
\begin{enumerate}
    \item \textbf{Naive users:}
    \begin{enumerate}
        \item Unsophisticated users, interact using the predefined interfaces.
        \item Typically fill forms and read reports generated by the database.
    \end{enumerate}

    \item \textbf{Application programmers:} Develop user interfaces by writing 
    application programs using various tools/frameworks.

    \item \textbf{Sophisticated users:}
    \begin{enumerate}
        \item Interact with the system without writing programs.
        \item Form requests using query languages or data analysis software, 
        mainly to explore the data in the database.
    \end{enumerate}

    \item \textbf{Database administrator (DBA):} Has central control over the 
    system. Functions are:
    \begin{enumerate}
        \item \textit{Schema definition:} Execute DDL to create the original 
        database schema.
        \item \textit{Storage structure and access-method definition:} Specify 
        parameters regarding physical organization of the data and generated indices.
        \item \textit{Schema and physical-organization modification:} Change 
        the schema and physical organization to suit changing needs of the 
        organization, or to improve performance.
        \item \textit{Granting authorization for data access:} Regulate user 
        access to appropriate parts of the database.
        \item \textit{Routine maintenance:} Periodic backup of data onto remote 
        servers, keeping a check on free disk space, monitoring jobs running on 
        the database to maintain performance.
    \end{enumerate}
\end{enumerate}

\end{document}